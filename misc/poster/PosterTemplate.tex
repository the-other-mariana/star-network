% UQ QOL/EQUS Poster format
% Forked from: https://www.overleaf.com/project/618cc965bcbc81513871601f
% Credit: Ali Furkan Kalay
% See: https://github.com/alfurka/gemini-uq
% Forked from
% https://rev.cs.uchicago.edu/k4rtik/gemini-uccs
% which is forked from
% https://github.com/anishathalye/gemini


\documentclass[final]{beamer}

% ====================
% Packages
% ====================
\usepackage[T1]{fontenc}
\usepackage{lmodern}
\usepackage[orientation=portrait,size=a0,scale=1.0]{beamerposter} % Current dimensions A0, put in your poster dimensions
\usetheme{gemini}
\usecolortheme{uchicago}
\usepackage{graphicx}
\usepackage{caption}
\usepackage{booktabs}
\usepackage{tikz}
\usepackage{pgfplots}
\pgfplotsset{compat=1.17}
\newcommand{\blu}{\color{blue}}
\usepackage{DejaVuSans}
%%%%%%%%%%%%%%%%%%%%%%%%%%%%%%%%%%%%%%%%%%%%%%%%%%%%%%%%%%%%%%%%%%%%%%%%%%%%%%
% Column environment setup
%%%%%%%%%%%%%%%%%%%%%%%%%%%%%%%%%%%%%%%%%%%%%%%%%%%%%%%%%%%%%%%%%%%%%%%%%%%%%%
% If you have N columns, choose \sepwidth and \colwidth such that
% (N+1)*\sepwidth + N*\colwidth = \paperwidth
% Follow structure to create difference column environments. 
\newlength{\sepwidthA} % Seperation distance between comulumns type A
\newlength{\colwidthA} % collumn width type A
\setlength{\sepwidthA}{0.25\paperwidth}
\setlength{\colwidthA}{0.5\paperwidth}

\newcommand{\separatorcolumnA}{\begin{column}{\sepwidthA}\end{column}}

% Second column environment. 
\newlength{\sepwidthB}
\newlength{\colwidthB}
\setlength{\sepwidthB}{0.0666\paperwidth}
\setlength{\colwidthB}{0.4\paperwidth}

\newcommand{\separatorcolumnB}{\begin{column}{\sepwidthB}\end{column}}

% You can also use these column commands to create columns inside columns and for creating new column formatting. 
% You can also have non even columns by creating more column environments or specifying the width when beginning a column environment. 
%%%%%%%%%%%%%%%%%%%%%%%%%%%%%%%%%%%%%%%%%%%%%%%%%%%%%%%%%%%%%%%%%%%%%%%%%%%%%%
% Title
%%%%%%%%%%%%%%%%%%%%%%%%%%%%%%%%%%%%%%%%%%%%%%%%%%%%%%%%%%%%%%%%%%%%%%%%%%%%%%

\title{\VeryHuge{Modelo de predicción de la tasa de pérdida de \\ 
paquetes en redes inalámbricas de sensores}}
\author{Daniel Heráclito Pérez Díaz \and Mariana Ávalos Arce}
\institute[shortinst]{Universidad Panamericana, Guadalajara, México}

%%%%%%%%%%%%%%%%%%%%%%%%%%%%%%%%%%%%%%%%%%%%%%%%%%%%%%%%%%%%%%%%%%%%%%%%%%%%%%
% Poster footer
%%%%%%%%%%%%%%%%%%%%%%%%%%%%%%%%%%%%%%%%%%%%%%%%%%%%%%%%%%%%%%%%%%%%%%%%%%%%%%

\footercontent{
\href{mailto:0190575@up.edu.mx}{0190575@up.edu.mx}
\href{mailto:0197495@up.edu.mx}{0197495@up.edu.mx} % this is a clickable link
  \hfill
  Especialidad en Ciencia de Datos \hfill
  {Poster Number: \#1} }
% (can be left out to remove footer)  

\begin{document}
%%%%%%%%%%%%%%%%%%%%%%%%%%%%%%%%%%%%%%%%%%%%%%%%%%%%%%%%%%%%%%%%%%%%%%%%%%%%%%
% Logo placements (optional)
%%%%%%%%%%%%%%%%%%%%%%%%%%%%%%%%%%%%%%%%%%%%%%%%%%%%%%%%%%%%%%%%%%%%%%%%%%%%%%
\addtobeamertemplate{headline}{}
{
    %\begin{tikzpicture}[remember picture,overlay] % Solid header bar
    \begin{tikzpicture}[remember picture,overlay,line width=\arrayrulewidth] % gradient header bar
      % UQ Reverse Logo 
      \node [anchor=north west, inner sep=3cm] at ([xshift=0cm,yshift=1.0cm]current page.north west)
      {\includegraphics[height=8cm]{logos/upwhite02.png}}; 
      % Logo 1, replace with custom logo
      \node [anchor=north east, inner sep=3cm] at ([xshift=0.0cm,yshift=2.0cm]current page.north east)
      {\includegraphics[height=8.0cm]{logos/tree.png}}; 
      % Extra logo 2
      %\node [anchor=north east, inner sep=3cm] at ([xshift=1.0cm,yshift=-10.0cm]current page.north east)
      %{\includegraphics[height=6.0cm]{logos/UQlockup-Purple-cmyk.eps}};
      %  Extra logo 3, or a QR Code
      \node [anchor=north west, inner sep=3cm] at ([xshift=-1.0cm,yshift=-13.0cm]current page.north west)
      {\includegraphics[height=7.0cm]{logos/qr-code.png}}; 
    \end{tikzpicture}
}

% ====================
% Body
% ====================

\begin{frame}[t]
%%%%%%%%%%%%%%%%%%%%%%%%%%%%%%%%%%%%%%%%%%%
%Section 1
%%%%%%%%%%%%%%%%%%%%%%%%%%%%%%%%%%%%%%%%%%%
\begin{columns}[t]
    \separatorcolumnA
    \begin{column}{\colwidthA}

        \begin{block}{Propuesta de la topología de red}
            \begin{figure}[!Ht]
            \centering
                \includegraphics[width=\linewidth]{Figures/NetworkDistribution_EM2022.png}
                \caption{Topología de Red.}
                \label{fig:top}
            \end{figure}
        \end{block}
    
    \end{column}

    \separatorcolumnA
\end{columns}

%%%%%%%%%%%%%%%%%%%%%%%%%%%%%%%%%%%%%%%%%%%
%Section 2 
%%%%%%%%%%%%%%%%%%%%%%%%%%%%%%%%%%%%%%%%%%%
\begin{columns}
%%%%%%%%%%%%%%%%%%%%%%%%%%%%%%%%%%%%%%%%%%%
%Section 2 column 1
%%%%%%%%%%%%%%%%%%%%%%%%%%%%%%%%%%%%%%%%%%%
\separatorcolumnB

    \begin{column}[T]{\colwidthB}

        \begin{block}{Metodología para encontrar la tendencia en los datos}

        \begin{figure}[!Ht]
            \centering
                \includegraphics[width=\linewidth]{Figures/pipeline.png}
                \caption{Esquema de la metodología a utilizar.}
                \label{fig:pipe}
        \end{figure}
    \end{block}
    
    \begin{block}{Hardware}
        El equipo o hardware que formarán la red a analizar está formado por placas de evaluación de la compañía Texas Instruments:

        \begin{figure}[!Ht]
            \centering
                \includegraphics[width=\linewidth]{Figures/hardware.png}
                \caption{Hardware de Texas Instruments.}
                \label{fig:hw}
        \end{figure}
    \end{block}

    \begin{alertblock}{La Red}
        Los protocolos implementados por la red están basados en el estándar IEEE 802.15.4.
    \end{alertblock}
        
\end{column}
\separatorcolumnB
%%%%%%%%%%%%%%%%%%%%%%%%%%%%%%%%%%%%%%%%%%%
%Section 2 column 2
%%%%%%%%%%%%%%%%%%%%%%%%%%%%%%%%%%%%%%%%%%%
\begin{column}[T]{\colwidthB}

    \begin{block}{Aplicaciones}
    Un desarrollo de IoT (Internet of Things) son las Ciudades Inteligentes o Smart Cities, donde éstas se definen como la mejora de la calidad de vida de los ciudadanos a través del uso de hardware, software, redes y datos. La pregunta que el modelo de predicción de la presente propuesta respondería sería principalmente si en una Smart City, cuya red presenta pérdida de paquetes, requiere de más routers o simplemente un reacomodo de sus nodos, a partir de la interferencia que presenta.

        \begin{figure}[!Ht]
            \centering
                \includegraphics[width=\linewidth]{Figures/smart-city-app.png}
                \caption{Smart City.}
                \label{fig:app}
        \end{figure}
    \end{block}

  

    

%%%%%%%%%%%%%%%%%%%%%%%%%%%%%%%%%%%%%%%%%%%%%%%%%%%%%%%%%%%%%%%%%%%%%%%%%%%%%%
% References
%%%%%%%%%%%%%%%%%%%%%%%%%%%%%%%%%%%%%%%%%%%%%%%%%%%%%%%%%%%%%%%%%%%%%%%%%%%%%%
\begin{block}{Referencias}

\nocite{*}
\bibliography{poster}% Produces the bibliography via BibTeX.
\bibliographystyle{plain}
\end{block} 

\end{column}
\separatorcolumnB
\end{columns}
\end{frame}
\end{document}
